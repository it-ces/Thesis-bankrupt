\documentclass[journal]{IEEEtai}

\usepackage[colorlinks,urlcolor=blue,linkcolor=blue,citecolor=blue]{hyperref}

\usepackage{color,array}
\usepackage{cite}
\usepackage{booktabs}% http://ctan.org/pkg/booktabs
\newcommand{\tabitem}{~~\llap{\textbullet}~~}
\usepackage{graphicx}
\usepackage{tikz}
\usepackage{caption}
\usepackage{comment}

%% \jvol{XX}
%% \jnum{XX}
%% \paper{1234567}
%% \pubyear{2020}
%% \publisheddate{xxxx 00, 0000}
%% \currentdate{xxxx 00, 0000}
%% \doiinfo{TQE.2020.Doi Number}

\newtheorem{theorem}{Theorem}
\newtheorem{lemma}{Lemma}
\setcounter{page}{1}
%% \setcounter{secnumdepth}{0}



\usepackage[backend=bibtex, style=ieee]{biblatex}
\addbibresource{library}

\begin{document}


\title{Bankruptcy prediction in Colombian case, using multilayer perceptron trained with memetic algorithm} 


\author{ Iván andrés Trujillo \IEEEmembership{PUJ MINTA}}

\author{Student:Iván Andrés Trujillo Abella,\\

\\
\\

Proffesors: Eliana María González Neira \& Gabriel Mauricio Zambrano Rey}
\maketitle

\begin{abstract}
Literature about Bankruptcy prediction is still incipient, therefore this work try to fill this gap by using machine learning  and metaheuristics techniques to find an optimal set of  weights in a MLP model.
\end{abstract}


\begin{IEEEkeywords}
Machine learning, Bankruptcy, Metaheuristics, Evolutionary Algorithms, Local Search, Memetic algorithms, Neural Networks, Multilayer Perceptron.
\end{IEEEkeywords}





\section{Results}


Data we retrieved the information of SuperIntendencia de Sociedades (in Spanish) for the period 2016-2019.

Classifying as bankrupt all firms  that enter a process of insolvency.




We consider all firms during the period 2016-2019 that enter that one year to another in a process, and uses financial ratios of one and two year before of the occurrence of the event, for instance firms that have a normal state in 2018 and in 2019 enter  in a process uses the financial ratios of 2018 and 2017 to model the event in 2019.

Were consider all satements that were presented in the december of each year, also were eliminated from data all events in 2016 of datasets to eliminated biased of financial ratios.

According  to the following table the number of events per year were:

According to the following Table 1:

Using shapiro-test was evaluated the normallity distribution of variables and T-test and kruskall wallis were carried out were performed to test if there is significative difference among bankrupt and no-bankrupt firms in one and two years.

\subsubsection{Exclusion criteria}
Were excluded from database those firms that are in preoperative process and firms that enter in default the initial base.	




\section{Benchmark}

The models used to benchmark were: Decision tree, logistic regression and multilayer perceptron.



\begin{table}
\thcenter
\begin{tabular}{lrrrrrr}
 & \multicolumn{2}{r}{Logistic Regression} & \multicolumn{2}{r}{Decision Tree  } & \multicolumn{2}{r}{Multilayer Perceptron} \\
 & Default & No-deafult & Default & No-default & No-default & no-default \\
precision & 1.000000 & 0.527027 & 0.789474 & 0.583333 & 0.000000 & 0.493671 \\
recall & 0.125000 & 1.000000 & 0.375000 & 0.897436 & 0.000000 & 1.000000 \\
f1-score & 0.222222 & 0.690265 & 0.508475 & 0.707071 & 0.000000 & 0.661017 \\
support & 40.000000 & 39.000000 & 40.000000 & 39.000000 & 40.000000 & 39.000000 \\
\end{tabular}
\end{table}




\section{Appendix}
The financial ratios used were:

\begin{itemize}
\item new
\end{itemize}

\newpage
\printbibliography





\end{document}